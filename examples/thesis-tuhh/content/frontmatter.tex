% Helper file that pulls subchapters together.

% Set initial pages to alpha, so that they do not collide with later Arabic numbering
% in the generated PDF. This won't show in print because page numbers aren't displayed
% until later. But you will be able to print the title page by printing page 'a', which
% would otherwise overlap with page '1', aka the first actual text page.
\pagenumbering{alph}
\maketitle% Print main title page
\subimport{frontmatter/}{colophon}

\frontmatter% In KOMAScript, resets pagenumber, uses Roman numerals etc.
%%%%%%%%%%%%%%%%%%%%%%%%%%%%%%%%%%%%%%%%%%%%%%%%%%%%%%%%
% Set bold chapters in bookmark to false
%%%%%%%%%%%%%%%%%%%%%%%%%%%%%%%%%%%%%%%%%%%%%%%%%%%%%%%%
\boldPDFchaptersfalse

% Note that \subincludefrom{}{} cannot be nested, therefore us \subimport
\subimport{frontmatter/}{task}
\subimport{frontmatter/}{authorship_declaration}
\subimport{frontmatter/}{abstract}

%%%%%%%%%%%%%%%%%%%%%%%%%%%%%%%%%%%%%%%%%%%%%%%%%%%%%%%%
% Table of Content
%%%%%%%%%%%%%%%%%%%%%%%%%%%%%%%%%%%%%%%%%%%%%%%%%%%%%%%%

\cleardoublepage % damit das Bookmark auf der richtigen Seite gesetzt wird
\pdfbookmark[1]{\contentsname}{toc} % Bookmark im PDF (hyperref-Paket)
\tableofcontents

%%%%%%%%%%%%%%%%%%%%%%%%%%%%%%%%%%%%%%%%%%%%%%%%%%%%%%%%
% All kind of listings
%%%%%%%%%%%%%%%%%%%%%%%%%%%%%%%%%%%%%%%%%%%%%%%%%%%%%%%%

\listoffigures%

\listoftables%

\listofexamples%

\listoflistings%

% If you want a list of reactions, uncomment the following line.
% Therefore you also have to switch on the own counter for reactions
% (own-counter=true) in the class file (where chemmacros is loaded).
% \listofreactions%

%%%%%%%%%%%%%%%%%%%%%%%%%%%%%%%%%%%%%%%%%%%%%%%%%%%%%%%%
% Set bold chapters in bookmark to true
%%%%%%%%%%%%%%%%%%%%%%%%%%%%%%%%%%%%%%%%%%%%%%%%%%%%%%%%
\boldPDFchapterstrue

%%%%%%%%%%%%%%%%%%%%%%%%%%%%%%%%%%%%%%%%%%%%%%%%%%%%%%%%
% Glossary with page list
%%%%%%%%%%%%%%%%%%%%%%%%%%%%%%%%%%%%%%%%%%%%%%%%%%%%%%%%
% addchap is KOMA equivalent for \chapter*, but also creates ToC entry, see also
% https://tex.stackexchange.com/a/116085/120853
% Use built-in macro \glossaryname for proper internationalization. With polyglossia, it
% will contain \text<language>{<glossary translation>}, which has been taken care of
% using \pdfstringdefDisableCommands{} in the class file
% October 2023: This is obsolet. The command isn't expanded or whatever and so
% it isn't printed bold. That's why our own translation is used. Here it works
% as in the past.
\addchap{\GetTranslation{Gloss}}%
\label{ch:glossary}

\emph{%
    \GetTranslation{GlossaryLegend}%
}%

% Print "unsorted" glossaries; these are in fact sorted, but externally using bib2gls.
% These will throw 'Token not allowed in PDF, removing \text<language>' warning.
% Specify title= manually if that gets too annoying.
\printunsrtglossary[
    type=symbols,
    style=symbunitlong,
]
\printunsrtglossary[
    type=numbers,
    style=numberlong,
]
\printunsrtglossary[
    type=subscripts,
    style=mcolalttree,
    nonumberlist,
]
\printunsrtglossary[
    type=abbreviations,
    style=long3colheader,
    title=\GetTranslation{Acronyms},
]

%%%%%%%%%%%%%%%%%%%%%%%%%%%%%%%%%%%%%%%%%%%%%%%%%%%%%%%%
% Glossary without page list
%%%%%%%%%%%%%%%%%%%%%%%%%%%%%%%%%%%%%%%%%%%%%%%%%%%%%%%%

\addchap{\GetTranslation{Gloss} without page lists}%

\emph{
    The following styles do not contain page lists of the entries' occurrences,
    leading to a cleaner, more concise look.
    Refer to the source code on how to achieve this (which options and styles to use).
}

% For all sorts of styles, see also
% https://www.dickimaw-books.com/gallery/glossaries-styles/

% Simply pass the `nonumberlist` parameter where desired/required:
\printunsrtglossary[
    type=symbols,
    style=symbunitlong,
    nonumberlist,
]
\printunsrtglossary[
    type=numbers,
    style=numberlong,
    nonumberlist,
]
\printunsrtglossary[
    type=subscripts,
    style=mcolalttree,
    nonumberlist,
]
\printunsrtglossary[
    type=abbreviations,
    % If `nonumberlist` is passed, the `long3colheader` style simply leaves the
    % corresponding table cells *empty* (leading to an entirely empty column), but does
    % not actually remove the column. So use a different, but equivalent style
    % altogether:
    style=longheader,
    % The `longheader` style prints the page list behind the description, just not in a
    % separate column. So also explicitly suppress the generation of that:
    nonumberlist,
    title=\GetTranslation{Acronyms},
]