% Multi-Language Document Example
% Demonstrates comprehensive multilingual support using babel

\documentclass[
    language=english,
    doctype=article,
    institution=none,
]{../../omnilatex}

% Configure polyglossia for multiple languages
\setdefaultlanguage{english}
\setotherlanguages{ngerman, french, spanish, italian, portuguese}

\RequirePackage{config/document-settings}
\addbibresource{bib/bibliography.bib}

% Title and author information
\title{\foreignlanguage{english}{Multi-Language Document Example}}
\author{\foreignlanguage{english}{OmniLaTeX Development Team}}

\begin{document}

% Set main language to English
\selectlanguage{english}

% Title page
\maketitle

% Table of contents
\tableofcontents

% English section
\section{\foreignlanguage{english}{Introduction}}

\foreignlanguage{english}{
    This document demonstrates the comprehensive multilingual capabilities of OmniLaTeX.
    The system supports multiple languages including English, German, French, Spanish,
    Italian, and Portuguese, ensuring proper typography and hyphenation for each language.
}

\foreignlanguage{english}{
    All text demonstrates font coverage and proper character rendering using the
    Libertinus Serif font family, which provides excellent support for European
    languages and special characters.
}

% German section
\section{\foreignlanguage{ngerman}{Deutsche Übersetzung}}

\foreignlanguage{ngerman}{
    Dies ist ein deutscher Text, der die Fähigkeiten des OmniLaTeX-Systems zur
    Verarbeitung mehrerer Sprachen demonstriert. Das System verwendet Babel für
    die korrekte Silbentrennung und Typografie.
}

\foreignlanguage{ngerman}{
    Deutsche Texte erfordern spezielle Zeichen wie Umlaute (ä, ö, ü) und das Eszett (ß).
    Diese Zeichen werden korrekt von der Libertinus-Schriftfamilie unterstützt.
}

% French section
\section{\foreignlanguage{french}{Traduction française}}

\foreignlanguage{french}{
    Ce document illustre les capacités multilingues d'OmniLaTeX. Le système
    supporte le français avec les caractères accentués appropriés (à, â, é, è, ê, ë,
    î, ï, ô, ù, û, ü, ÿ) et la typographie française correcte.
}

\foreignlanguage{french}{
    La magnifique typographie de la Libertinus Serif assure une lecture parfaite
    de tous les textes multilingues dans ce document d'exemple.
}

% Spanish section
\section{\foreignlanguage{spanish}{Traducción española}}

\foreignlanguage{spanish}{
    Este documento demuestra las capacidades multilingües del sistema OmniLaTeX.
    El español incluye caracteres especiales como la ñ y los acentos (á, é, í, ó, ú),
    que son correctamente renderizados con la familia tipográfica Libertinus.
}

\foreignlanguage{spanish}{
    El soporte tipográfico completo garantiza que todos los idiomas europeos
    se vean perfectos en cualquier documento generado con OmniLaTeX.
}

% Italian section
\section{\foreignlanguage{italian}{Traduzione italiana}}

\foreignlanguage{italian}{
    Questo documento dimostra le capacità multilingue del sistema OmniLaTeX.
    La lingua italiana utilizza caratteri accentati (à, è, é, ì, í, î, ò, ó, ù, ú, û, ü)
    che vengono correttamente supportati dal font Libertinus Serif.
}

\foreignlanguage{italian}{
    L'eccellente supporto tipografico assicura che tutti i testi multilingue
    siano perfettamente leggibili e esteticamente gradevoli.
}

% Portuguese section
\section{\foreignlanguage{portuguese}{Tradução portuguesa}}

\foreignlanguage{portuguese}{
    Este documento demonstra as capacidades multilingues do sistema OmniLaTeX.
    O português inclui caracteres especiais como a cedilha (ç) e acentos (á, à, â, ã,
    é, ê, í, ó, ô, õ, ú, û) que são corretamente renderizados.
}

\foreignlanguage{portuguese}{
    O suporte tipográfico completo da família Libertinus garante uma
    apresentação perfeita para todos os idiomas supported.
}

% Mixed language example
\section{\foreignlanguage{english}{Mixed Language Content}}

\foreignlanguage{english}{
    This section demonstrates mixing languages within the same document.
    For example, we can discuss \emph{\foreignlanguage{french}{la typographie française}}
    and \emph{\foreignlanguage{german}{deutsche Silbentrennung}} in the same paragraph.
    \foreignlanguage{spanish}{Esto demuestra} la flexibilidad del sistema OmniLaTeX
    \foreignlanguage{italian}{per la gestione} \foreignlanguage{portuguese}{de múltiplos idiomas}
    \foreignlanguage{english}{in a single document}.
}

% Mathematical content with international examples
\section{\foreignlanguage{english}{Mathematical Content}}

\foreignlanguage{english}{
    Mathematics transcends language barriers. The formula for the area of a circle
    remains universal: $A = \pi r^2$. However, we can provide examples in different languages:
}

\begin{equation}
    \text{\foreignlanguage{german}{Fläche des Kreises: }} A = \pi r^{2}
\end{equation}

\begin{equation}
    \text{\foreignlanguage{french}{Aire du cercle: }} A = \pi r^2
\end{equation}

\begin{equation}
    \text{\foreignlanguage{spanish}{Área del círculo: }} A = \pi r^2
\end{equation}

\begin{equation}
    \text{\foreignlanguage{italian}{Area del cerchio: }} A = \pi r^2
\end{equation}

\begin{equation}
    \text{\foreignlanguage{portuguese}{Área do círculo: }} A = \pi r^2
\end{equation}

% Bibliography example
\section{\foreignlanguage{english}{Bibliography}}

\foreignlanguage{english}{
    This document demonstrates font coverage for multiple languages. The bibliography
    system properly handles international characters and provides appropriate sorting
    according to language-specific rules.
}

\nocite{*}
\printbibliography

\end{document}