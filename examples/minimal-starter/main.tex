% Minimal OmniLaTeX Starter Template
% This file demonstrates all major features idiomatically with minimal content

%!TEX TS-program = lualatex

\documentclass[
    language=english,
    institution=none,        % Generic (no institution branding)
    oneside,                 % Single-sided (screen viewing)
    % loadGlossaries,          % Enable glossaries/acronyms - temporarily disabled
]{../../omnilatex}

% Document settings
\RequirePackage{config/document-settings}

% Bibliography (local to this example)
\addbibresource{bib/bibliography.bib}

\begin{document}

% Title page
\maketitle

% Table of contents
\tableofcontents

\chapter{Introduction}

This is a minimal starter template demonstrating all major OmniLaTeX capabilities.
Each section shows one feature with minimal but idiomatic usage.

\section{Text Formatting}

Basic text with \textbf{bold}, \textit{italic}, and \texttt{monospace} fonts.
Unicode support: αβγ, ℝℂℕ, ∀∃∈∋.

\section{Citations}

Reference bibliography entries with \verb|\cite{}|: \cite{knuth1984texbook}.
Multiple citations: \cite{lamport1994latex,goossens1994latex}.

\section{Cross-References}

Reference sections with \verb|\ref{}|: See Section~\ref{sec:lists}.
Reference figures: Figure~\ref{fig:example}.
Reference tables: Table~\ref{tab:example}.

\chapter{Lists and Structures}
\label{sec:lists}

\section{Unordered Lists}

\begin{itemize}
    \item First item
    \item Second item
          \begin{itemize}
              \item Nested item
              \item Another nested item
          \end{itemize}
    \item Third item
\end{itemize}

\section{Ordered Lists}

\begin{enumerate}
    \item First step
    \item Second step
    \item Third step
\end{enumerate}

\section{Description Lists}

\begin{description}
    \item[OmniLaTeX] Universal LaTeX template
    \item[LuaLaTeX] Modern LaTeX engine
    \item[KOMA-Script] Document class collection
\end{description}

\chapter{Mathematics}

\section{Inline Math}

Einstein's famous equation: $E = mc^2$.
Pythagorean theorem: $a^2 + b^2 = c^2$.

\section{Display Math}

The quadratic formula:
\begin{equation}
    x = \frac{-b \pm \sqrt{b^2 - 4ac}}{2a}
    \label{eq:quadratic}
\end{equation}

Maxwell's equations:
\begin{align}
    \nabla \cdot \mathbf{E}  & = \frac{\rho}{\varepsilon_0}                                                   \\
    \nabla \cdot \mathbf{B}  & = 0                                                                            \\
    \nabla \times \mathbf{E} & = -\frac{\partial \mathbf{B}}{\partial t}                                      \\
    \nabla \times \mathbf{B} & = \mu_0 \mathbf{J} + \mu_0\varepsilon_0 \frac{\partial \mathbf{E}}{\partial t}
\end{align}

Reference equations: Equation~\eqref{eq:quadratic}.

\chapter{Figures and Tables}

\section{Figures}

\begin{figure}[htbp]
    \centering
    % Example with TikZ - create a simple diagram
    \begin{tikzpicture}
        \draw[thick, ->] (0,0) -- (3,0) node[right] {$x$};
        \draw[thick, ->] (0,0) -- (0,3) node[above] {$y$};
        \draw[blue, thick, domain=0:2.5] plot (\x, {\x*\x});
        \node at (1.5,3) {$f(x) = x^2$};
    \end{tikzpicture}
    \caption{Example TikZ figure showing $f(x) = x^2$}
    \label{fig:example}
\end{figure}

% For external images, use:
% \begin{figure}[htbp]
%     \centering
%     \includegraphics[width=0.8\textwidth]{../../assets/yourimage.pdf}
%     \caption{Your caption}
%     \label{fig:yourimage}
% \end{figure}

\section{Tables}

\begin{table}[htbp]
    \centering
    \caption{Example table with data}
    \label{tab:example}
    \begin{tabular}{lcc}
        \toprule
        \textbf{Item} & \textbf{Value 1} & \textbf{Value 2} \\
        \midrule
        First         & 10.5             & 20.3             \\
        Second        & 15.2             & 18.7             \\
        Third         & 12.8             & 22.1             \\
        \bottomrule
    \end{tabular}
\end{table}

\chapter{Code Listings}

\section{Python Example}

\begin{listing}[htbp]
    \begin{minted}{python}
def fibonacci(n):
    """Calculate the nth Fibonacci number."""
    if n <= 1:
        return n
    return fibonacci(n-1) + fibonacci(n-2)

print(fibonacci(10))  # Output: 55
\end{minted}
    \caption{Recursive Fibonacci function in Python}
    \label{lst:fibonacci}
\end{listing}

\section{LaTeX Example}

\begin{listing}[htbp]
    \begin{minted}{latex}
\documentclass{article}
\begin{document}
Hello, \LaTeX{}!
\end{document}
\end{minted}
    \caption{Minimal LaTeX document}
\end{listing}

% For inline code, use: \mintinline{python}{def foo(): pass}
% Or for simple code: \texttt{inline\_code}

\chapter{Glossaries and Acronyms}

\section{Using Glossaries}

% First use shows full definition
This template uses \gls{latex} for typesetting.
The \gls{tuhh} is a research university in Hamburg.

% Subsequent uses show abbreviated form
Using \gls{latex} and \gls{tuhh} again.

% Force full or short forms
\Glspl{cpu} are essential for computing.  % Plural
\acrshort{cpu} is the short form only.
\acrlong{cpu} is the long form only.

\section{Adding New Terms}

% Add terms to: ../../bib/glossaries/abbreviations.bib
% Format:
% @abbreviation{key,
%     short = {SHORT},
%     long  = {Long form},
% }

\chapter{Advanced Features}

\section{Boxes and Callouts}

% Example box (requires custom definition or package)
\begin{center}
    \fbox{\parbox{0.8\textwidth}{%
            \textbf{Note:} This is an example callout box.
            You can add important information here.
        }}
\end{center}

\section{TODO Notes}

% Requires todonotes option in documentclass
% \todo{This is a TODO note visible in draft mode}
% \todo[inline]{This is an inline TODO note}

\section{Hyperlinks}

External link: \url{https://www.latex-project.org/}

Link with text: \href{https://www.ctan.org}{CTAN Package Repository}

Email: \href{mailto:example@example.com}{example@example.com}

\section{Footnotes}

This text has a footnote\footnote{This is the footnote text.}.
Multiple footnotes\footnote{First footnote.} are possible\footnote{Second footnote.}.

\section{Colors}

Text can be \textcolor{red}{red}, \textcolor{blue}{blue}, or \textcolor{green}{green}.

You can define custom colors in your document or institution configuration.

\chapter{Bibliography and References}

All citations appear automatically in the bibliography.
Add entries to \texttt{../../bib/bibliography.bib}.

Example BibTeX entry:
\begin{verbatim}
@book{knuth1984texbook,
    author = {Donald E. Knuth},
    title = {The TeXbook},
    year = {1984},
    publisher = {Addison-Wesley}
}
\end{verbatim}

% Bibliography is printed automatically at the end if using backmatter
% Or manually with: \printbibliography

\appendix

\chapter{Appendix Example}

Appendices use the same chapter structure.
They're automatically labeled A, B, C, etc.

\section{Additional Data}

Extra information that doesn't fit in the main text.

\section{Supplementary Code}

\begin{minted}{bash}
#!/bin/bash
# Build script example
latexmk -pdf main.tex
\end{minted}

% Print glossaries
\printglossaries

% Print bibliography
\printbibliography[heading=bibintoc]

\end{document}