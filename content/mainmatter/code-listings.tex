\chapter{Code Syntax Highlighting}
\label{ch:code-listings}

To properly typeset code in \LaTeX{}, we use the \ctanpackage{minted} package.
It relies on Python, and calls in outside help for syntax highlighting using Python's
\texttt{pygments} package.
As such, it requires \texttt{--shell-escape} to compile, and of course Python.
The latter can be a pain in the buttocks to set up; Docker usage is especially useful here.
Since \texttt{pygments} has nothing to do with \LaTeX{} (whose ecosystem is a sad mess),
but is instead a regular old Python package, chances are the language of your choice is
not only available but also well\-/supported!%
\footnote{%
    For example, \ctanpackage{listings}, the inferior alternative to \ctanpackage{minted},
    still had no Python 3 support (only basic 2) in 2020, at the time of originally writing this.
    At that point, Python 3 was over 10 years old already and Python 2 was end-of-life.
}

\paragraph{Color Scheme}
Note how the color scheme is the same throughout languages.
The idea is that keywords of similar importance, status or semantics are highlighted uniformly.
For example, keywords for class and function definitions, like \mintinline{python}{class}
for Python or \mintinline{matlab}{classdef} for MATLAB.%
\footnote{%
Notice how these keywords were created using an \emph{inline} listing, like:
\mintinline{python}{y = [file_patterns[x] for x in ["send", "help"]]}.
}
Another example are error\-/handling and -throwing keywords, which many languages
offer.
All these different types should be identified and treated equally.
\ctanpackage{minted} is a widely used package and knows about a lot of languages.
You can check it out at
\begin{center}
    \url{https://pygments.org/demo/}.
\end{center}

If the current colors are not to your liking they can be changed easily using
\ctanpackage{minted}'s \texttt{style} option.
Refer to the comments in the source code on how to see available styles.

\paragraph{References}
Individual code lines can also be referenced.
For example, we find a \mintinline{python}{return} statement on
% NOTE: referencing single "minted" lines *CANNOT* be done with cleveref, unfortunately.
% See: https://tex.stackexchange.com/q/132420/120853 . A plain old `ref` does the trick.
line \ref{line:python_return} in \cref{lst:float_example}.
That line is also highlighted, using \ctanpackage{minted}'s \texttt{highlightlines} option.

\section{Python}
The following examples are not always complete or functioning, they are only supposed
to showcase the available syntax highlighting.
The base style looks like:
\begin{minted}{python}
    def get_nonempty_line(
        lines: Iterable[str],
        last: bool = True
    ) -> str:
        if last:
            lines = reversed(lines)
        return next(line for line in lines if line.rstrip())
\end{minted}
It is intended for (small) samples of code that flow into the surrounding text.
A second feature are code listings as regular floats, as demonstrated in
\cref{lst:float_example}.
As floats, they behave like any other figure, table \iecfeg{etc}.

\begin{listing}
    \caption{%
        This is a caption.
        Listings cannot be overly long since floats do not page-break. Therefore is the longlisting environment.% No period here!
    }
    \label{lst:float_example}
    \begin{minted}[highlightlines={24}]{python}
        import json
        import logging.config
        from pathlib import Path

        from resources.helpers import path_relative_to_caller_file

        ¬\phstring{\LaTeX{} can go in here: \(\sum_{i = 1}^{n} a + \frac{\pi}{2} \)}¬

        def set_up_logging(logger_name: str) -> logging.Logger:
            """Set up a logging configuration."""
            config_filepath = path_relative_to_caller_file("logger.json")  # same directory

            try:
                with open(config_filepath) as config_file:
                    config: dict = json.load(config_file)
                logging.config.dictConfig(config)
            except FileNotFoundError:
                logging.basicConfig(
                    level=logging.INFO, format="[%(asctime)s: %(levelname)s] %(message)s"
                )
                logging.warning(f"Using fallback: no logging config found at {config_filepath}")
                logger_name = __name__

            return logging.getLogger(logger_name) ¬\label{line:python_return}¬
    \end{minted}
\end{listing}

\Cref{lst:longlisting_example} showcases breaking across pages, imported from a file, probably best suited for an appendix:

\begin{longlisting}
    \caption{%
        This is a page breaking code, with a caption, which is imported from a file!%
    }
    \label{lst:longlisting_example}
    \inputminted{python}{assets/code/ansi_escaped_string.py}
\end{longlisting}

\section{MATLAB}

This section contains example code for MATLAB, for example:
\begin{minted}{matlab}
    %{
        Universal Gas Constant for SIMULINK.
    %}
    R_m = Simulink.Parameter;
        R_m.Value = 8.3144598;
        R_m.Description = 'universal gas constant';
        R_m.DocUnits = 'J/(mol*K)';
\end{minted}
Of course, floats (see \cref{lst:matlab_class_definition}) are available as well.
So are longer sections, like \cref{lst:matlab_longlisting}.

\begin{listing}
    \caption[A class definition in MATLAB]{%
        A class definition in MATLAB, from \cite{mathworksCreateSimpleClass2020}%
    }
    \label{lst:matlab_class_definition}
    \begin{minted}{matlab}
        classdef BasicClass
            properties
                Value {mustBeNumeric}
            end
            methods
                function r = roundOff(obj)
                    r = round([obj.Value],2);
                end
                function r = multiplyBy(obj,n)
                    r = [obj.Value] * n;
                end
            end
        end
    \end{minted}
\end{listing}

\begin{longlisting}
    \caption{%
        This is a page breaking matlab code, with a caption, which is imported from a file!%
    }
    \label{lst:matlab_longlisting}
    \inputminted{python}{assets/code/matlab_long.m}
\end{longlisting}

\subsection{MATLAB/Simulink icons}

For an older project, MATLAB/Simulink vector icons were created.
They are included here at the off\-/chance that someone else might find a use for these.
\begin{itemize}
    \item \mtlbsmlkicon{matlab_object_box}
    \item \mtlbsmlkicon{matlab_struct}
    \item \mtlbsmlkicon{matlab_table}
    \item \mtlbsmlkicon{simulink_algebraic_constraint}
    \item \mtlbsmlkicon{simulink_base_workspace}
    \item \mtlbsmlkicon{simulink_configuration}
    \item \mtlbsmlkicon{simulink_data_dictionary}
    \item \mtlbsmlkicon{simulink_library_model}
    \item \mtlbsmlkicon{simulink_library}
    \item \mtlbsmlkicon{simulink_log_data}
    \item \mtlbsmlkicon{simulink_lut}
    \item \mtlbsmlkicon{simulink_model_workspace}
    \item \mtlbsmlkicon{simulink_model}
    \item \mtlbsmlkicon{simulink_referenced_model}
    \item \mtlbsmlkicon{simulink_step_block}
\end{itemize}

\section{Modelica}

Naturally, floating and all other environments and styles are also available for
Modelica.
The syntax highlighting for a few basic code samples \autocite{wikipediacontributorsModelica2021} looks like:

\begin{minted}{modelica}
    x := 2 + y;
    x + y = 3 * z;

    model FirstOrder
        parameter Real c=1 "Time constant";
        Real x (start=10) "An unknown";
    equation
        der(x) = -c*x "A first order differential equation";
    end FirstOrder;

    type Voltage = Real(quantity="ElectricalPotential", unit="V");
    type Current = Real(quantity="ElectricalCurrent", unit="A");

    connector Pin "Electrical pin"
        Voltage      v "Potential at the pin";
        flow Current i "Current flowing into the component";
    end Pin;

    model Capacitor
        parameter Capacitance C;
        Voltage u "Voltage drop between pin_p and pin_n";
        Pin pin_p, pin_n;
    equation
        0 = pin_p.i + pin_n.i;
        u = pin_p.v - pin_n.v;
        C * der(u) = pin_p.i;
    end Capacitor;

    model SignalVoltage
        "Generic voltage source using the input signal as source voltage"
        Interfaces.PositivePin p;
        Interfaces.NegativePin n;
        Modelica.Blocks.Interfaces.RealInput v(unit="V")
            "Voltage between pin p and n (= p.v - n.v) as input signal";
        SI.Current i "Current flowing from pin p to pin n";
    equation
        v = p.v - n.v;
        0 = p.i + n.i;
        i = p.i;
    end SignalVoltage;

    model Circuit
        Capacitor C1(C=1e-4) "A Capacitor instance from the model above";
        Capacitor C2(C=1e-5) "A Capacitor instance from the model above";
            ...
    equation
        connect(C1.pin_p, C2.pin_n);
            ...
    end Circuit;
\end{minted}

\section{Lua}

Files can also be read into \LaTeX{} directly.
For example, the following is some current Lua code \emph{used for this very document}:
\inputminted{lua}{./lua/git-metadata.lua}
